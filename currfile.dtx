% \iffalse meta-comment
%% Copyright (c) 2010 by Martin Scharrer <martin@scharrer-online.de>
%% -----------------------------------------------------------------
%%
%% This work may be distributed and/or modified under the
%% conditions of the LaTeX Project Public License, either version 1.3c
%% of this license or (at your option) any later version.
%% The latest version of this license is in
%%
%%   http://www.latex-project.org/lppl.txt
%%
%% and version 1.3c or later is part of all distributions of LaTeX
%% version 2008/05/04 or later.
%%
%% This work has the LPPL maintenance status `maintained'.
%%
%% The Current Maintainer of this work is Martin Scharrer.
%%
%% This work consists of the files currfile.dtx, currfile.ins
%% and the derived file currfile.sty.
%%
%% $Id: standalone.dtx 1803 2010-04-03 13:59:44Z martin $
% \fi
%
% \iffalse
%<*driver>
\ProvidesFile{currfile.dtx}
%</driver>
%<package>\NeedsTeXFormat{LaTeX2e}[1999/12/01]
%<package>\ProvidesPackage{currfile}
%<*package>
    [2010/04/08 v0.1 Current input file name and path]
%</package>
%
%<*driver>
\documentclass{ltxdoc}
\usepackage{currfile}[2010/04/08]
\usepackage{ifpdf}
\usepackage{hyperref}
\EnableCrossrefs
\CodelineIndex
\RecordChanges
\providecommand*\pkg{\texttt}
\begin{document}
  \DocInput{currfile.dtx}
  \PrintChanges
  \PrintIndex
\end{document}
%</driver>
% \fi
%
% \CheckSum{148}
%
% \CharacterTable
%  {Upper-case    \A\B\C\D\E\F\G\H\I\J\K\L\M\N\O\P\Q\R\S\T\U\V\W\X\Y\Z
%   Lower-case    \a\b\c\d\e\f\g\h\i\j\k\l\m\n\o\p\q\r\s\t\u\v\w\x\y\z
%   Digits        \0\1\2\3\4\5\6\7\8\9
%   Exclamation   \!     Double quote  \"     Hash (number) \#
%   Dollar        \$     Percent       \%     Ampersand     \&
%   Acute accent  \'     Left paren    \(     Right paren   \)
%   Asterisk      \*     Plus          \+     Comma         \,
%   Minus         \-     Point         \.     Solidus       \/
%   Colon         \:     Semicolon     \;     Less than     \<
%   Equals        \=     Greater than  \>     Question mark \?
%   Commercial at \@     Left bracket  \[     Backslash     \\
%   Right bracket \]     Circumflex    \^     Underscore    \_
%   Grave accent  \`     Left brace    \{     Vertical bar  \|
%   Right brace   \}     Tilde         \~}
%
%
% \changes{v0.1}{2010/04/08}{Initial version}
%
% \GetFileInfo{currfile.dtx}
%
% \DoNotIndex{\newcommand,\newenvironment}
% 
% \ifpdf
% \hypersetup{%
%   pdfauthor   = {Martin Scharrer <martin@scharrer-online.de>},
%   pdftitle    = {The currfile package},
%   pdfsubject  = {Documentation of LaTeX package 'currfile'},
%   pdfkeywords = {current filename, filename, file, name, LaTeX, TeX}
% }%
% \fi
% \clearpage
% \null
% \vspace*{-2em}
% \begin{center}
%   {\LARGE\sffamily The \emph{currfile} Package\\[\medskipamount]}
%   {\large Martin Scharrer \\[\medskipamount]\normalsize 
%   \url{martin@scharrer-online.de}\\[.8ex]
%   \url{http://www.ctan.org/pkg/standalone/}\\[\bigskipamount]}
%   {\large Version \fileversion\ -- \filedate}\\
% \end{center}
% \vspace{1.2em}%
%
% \begin{abstract}
% This small package provides the file name and path information of the current input file as \LaTeX{} macros.
% \end{abstract}
%
% \section{Usage}
%
% \DescribeMacro{\currfiledir}
% \DescribeMacro{\currfilebase}
% \DescribeMacro{\currfileext}
% \DescribeMacro{\currfilename}
% \DescribeMacro{\currfilepath}
% The directory, base (name without extension), extension (without dot), name (=base+`|.|'+ext) and path (=dir+name) of the current file are provided by these macros.
% This means that the macros returns the file information of the file they are used in. All macros are fully expanded, i.e.\ only hold
% text and not further macros.
% Special care is taken to keep the file information of |\include|d files till the final |\clearpage| command, so that page header and footer of the last page 
% will hold the correct data.
%
% Only files loaded with the \LaTeX{} macros |\input| or |\include| and the main file are taken into account. Files loaded using package macros like |\usepackage| or internal
% macros \cs{@input} and \cs{@@input} are ignored.
%
% More detailed information can be found in the implementation section (\ref{sec:impl}) if required.
%
% \subsection*{Package Options}
% The package provides two options |mainext| and |maindir| which can be used to provide the extension and directory of the main file.
% This is required if the above macros should be used for the main file itself and if this does has a file extension other than `|.tex|' (e.g.\ a |.dtx| file)
% or is not located in the current directory.
%
% \section{Similar Packages}
% The |fink| package (\emph{fi}le \emph{n}ame \emph{k}eeper) provides a similar functionality. It has inspired this package in several points (e.g.\ package
% options). However, it does not exclude package and other preamble files and does not take care to change the filename \emph{after} the |\clearpage| of
% |\include|.
%
% \StopEventually{}
%
%
% \section{Implementation}\label{sec:impl}
%
% \iffalse
%<*package>
% \fi
% \subsection{Options}
%
%    \begin{macrocode}
\RequirePackage{kvoptions}
\SetupKeyvalOptions{family=currfile,prefix=currfile@}
\DeclareStringOption[tex]{mainext}
\DeclareStringOption[]{maindir}
\ProcessKeyvalOptions*\relax
%    \end{macrocode}
%
%
% \subsection{File Hooks}
% The \pkg{filehook} package is used to execute the macros at the correct places.
% The internal, not user, interface is used to make sure that the file names are valid for all other hooks.
%
%    \begin{macrocode}
\RequirePackage{filehook}
%    \end{macrocode}
%
%    \begin{macrocode}
\filehook@prefixwarg\filehook@include@atbegin{%
  \currfile@push
  \currfile@set{#1}%
}
\filehook@appendwarg\filehook@include@after{%
  \currfile@pop
}
\filehook@prefixwarg\filehook@input@atbegin{%
  \currfile@push
  \currfile@set{#1}%
}
\filehook@appendwarg\filehook@input@atend{%
  \currfile@pop
}
%    \end{macrocode}
%
%
% \subsection{Set Current Values}
%
% \begin{macro}{\currfile@set}
% Sets the file information which are parsed by \LaTeX's |\filename@parse|.
%    \begin{macrocode}
\def\currfile@set#1{%
  \begingroup
    \filename@parse{#1}%
    \global\let\currfiledir\filename@area
    \global\let\currfilebase\filename@base
    \xdef\currfileext{\ifx\filename@ext\relax tex\else\filename@ext\fi}%
    \xdef\currfilename{\currfilebase\ifx\currfileext\empty\else.\currfileext\fi}%
    \xdef\currfilepath{\currfiledir\currfilename}%
  \endgroup
%<debug> \expandafter\gdef\expandafter\dindent\expandafter{\dindent\space}%
%<debug> \message{^^JDEBUG: \dindent\empty Entering file '\currfilename' ^^J }%
}
%    \end{macrocode}
% \end{macro}
%
%
% \subsection{File Stack}
% The file information are pushed and popped on a stack to save and restore them when
% entering and leaving a sub-file, respectively. This is quite similar to the way 
% \LaTeX\ saves file base names and extension as well as the `@' status (letter or other) for
% package and class files.
%
% \begin{macro}{\currfile@push}
%    \begin{macrocode}
\def\currfile@push{%
  \xdef\currfile@stack{%
    {\currfiledir}%
    {\currfilebase}%
    {\currfileext}%
    \currfile@stack
  }%
}
%    \end{macrocode}
% \end{macro}
%
% \begin{macro}{\currfile@pop}
%    \begin{macrocode}
\def\currfile@pop{%
%<debug> \message{^^JDEBUG: \dindent\empty Leaving file '\currfilename' ^^J }%
  \ifx\currfile@stack\empty
    \PackageWarning{currfile}{File stack underflow!}{}{}%
    \global\let\currfile@stack\currfile@stackinit
  \fi
  \expandafter\currfile@pop@\currfile@stack\relax
  \relax\relax\relax
%<debug> \message{^^JDEBUG: \dindent\empty Restoring file '\currfilename' ^^J }%
}
%    \end{macrocode}
% \end{macro}
%
% \begin{macro}{\currfile@pop@}
%    \begin{macrocode}
\def\currfile@pop@#1#2#3#4\relax{%
  \gdef\currfiledir{#1}%
  \gdef\currfilebase{#2}%
  \gdef\currfileext{#3}%
  \xdef\currfilename{\currfilebase\ifx\currfileext\empty\else.\currfileext\fi}%
  \xdef\currfilepath{\currfiledir\currfilename}%
  \gdef\currfile@stack{#4}%
%<debug> \expandafter\expandafter\expandafter\gdef
%<debug> \expandafter\expandafter\expandafter\dindent
%<debug> \expandafter\expandafter\expandafter{\expandafter\@gobble\dindent}%
}
%    \end{macrocode}
% \end{macro}
%
% \iffalse
%<debug> \def\dindent{}
% \fi
%
% \begin{macro}{\currfile@stack}
% \begin{macro}{\currfile@stackinit}
% Place |\jobname| values on stack and use this as init value.
%    \begin{macrocode}
\def\currfile@stack{}
\currfile@set{\currfile@maindir\jobname.\currfile@mainext}
\currfile@push
\let\currfile@stackinit\currfile@stack
%    \end{macrocode}
% \end{macro}
% \end{macro}
%
% \iffalse
%</package>
% \fi
% \Finale
\endinput
